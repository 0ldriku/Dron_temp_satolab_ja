% ****************************************************************************************************
% classicthesis-config.tex
% formerly known as loadpackages.sty, classicthesis-ldpkg.sty, and classicthesis-preamble.sty
% Use it at the beginning of your ClassicThesis.tex, or as a LaTeX Preamble
% in your ClassicThesis.{tex,lyx} with % ****************************************************************************************************
% classicthesis-config.tex
% formerly known as loadpackages.sty, classicthesis-ldpkg.sty, and classicthesis-preamble.sty
% Use it at the beginning of your ClassicThesis.tex, or as a LaTeX Preamble
% in your ClassicThesis.{tex,lyx} with % ****************************************************************************************************
% classicthesis-config.tex
% formerly known as loadpackages.sty, classicthesis-ldpkg.sty, and classicthesis-preamble.sty
% Use it at the beginning of your ClassicThesis.tex, or as a LaTeX Preamble
% in your ClassicThesis.{tex,lyx} with % ****************************************************************************************************
% classicthesis-config.tex
% formerly known as loadpackages.sty, classicthesis-ldpkg.sty, and classicthesis-preamble.sty
% Use it at the beginning of your ClassicThesis.tex, or as a LaTeX Preamble
% in your ClassicThesis.{tex,lyx} with \input{classicthesis-config}
% ****************************************************************************************************
% If you like the classicthesis, then I would appreciate a postcard.
% My address can be found in the file ClassicThesis.pdf. A collection
% of the postcards I received so far is available online at
% http://postcards.miede.de
% ****************************************************************************************************


% ****************************************************************************************************
% 0. Set the encoding of your files. UTF-8 is the only sensible encoding nowadays. If you can't read
% äöüßáéçèê∂åëæƒÏ€ then change the encoding setting in your editor, not the line below. If your editor
% does not support utf8 use another editor!
% ****************************************************************************************************
\usepackage{iftex}
\ifPDFTeX
  \PassOptionsToPackage{utf8}{inputenc}
  \usepackage{inputenc}
  \PassOptionsToPackage{T1}{fontenc} % T2A for cyrillics
  \usepackage{fontenc}
\else
  \usepackage{fontspec}
\fi


% ****************************************************************************************************
% 1. Configure classicthesis for your needs here, e.g., remove "drafting" below
% in order to deactivate the time-stamp on the pages
% (see ClassicThesis.pdf for more information):
% ****************************************************************************************************
\PassOptionsToPackage{
  drafting=false,    % print version information on the bottom of the pages
  tocaligned=false, % the left column of the toc will be aligned (no indentation)
  dottedtoc=false,  % page numbers in ToC flushed right
  eulerchapternumbers=false, % use AMS Euler for chapter font (otherwise Palatino)
  floatperchapter=true,     % numbering per chapter for all floats (i.e., Figure 1.1)
  eulermath=false,  % use awesome Euler fonts for mathematical formulae (only with pdfLaTeX)
  beramono=true,    % toggle a nice monospaced font (w/ bold)
  palatino=false,    % deactivate standard font for loading another one, see the last section at the end of this file for suggestions
  %linedheaders=true, % obsolete / available for backwards compatibility
  style=classicthesis % classicthesis, arsclassica, linedheaders, plain
}{classicthesis}

% ****************************************************************************************************
% 2. Personal data and user ad-hoc commands (insert your own data here)
% ****************************************************************************************************
% ****************************************************************************************************
% 2. Personal data and user ad-hoc commands (insert your own data here)
% ****************************************************************************************************
\newcommand{\myTitle}{The Title of the Thesis\xspace}
\newcommand{\mySubtitle}{The Subtitle of the Thesis\xspace}
\newcommand{\myDegree}{Your Degree\xspace}
\newcommand{\myName}{Your Name\xspace}
\newcommand{\myProf}{Your Supervisor\xspace}
\newcommand{\myOtherProf}{Your Other Supervisor\xspace}
\newcommand{\mySupervisor}{Your Supervisor\xspace}
\newcommand{\myFaculty}{Your Faculty\xspace}
\newcommand{\myDepartment}{Your Department\xspace}
\newcommand{\myUni}{Your Uni\xspace}
\newcommand{\myLocation}{Location\xspace}
\newcommand{\myTime}{Month Year\xspace}
\newcommand{\myVersion}{v1.0\xspace}

% ********************************************************************
% Setup, finetuning, and useful commands
% ********************************************************************
\providecommand{\mLyX}{L\kern-.1667em\lower.25em\hbox{Y}\kern-.125emX\@}
\newcommand{\ie}{i.\,e.}
\newcommand{\Ie}{I.\,e.}
\newcommand{\eg}{e.\,g.}
\newcommand{\Eg}{E.\,g.}
\newcommand{\autorefja}[1]{%
  \begingroup
  \crefname{chapter}{第}{第}%
  \crefformat{chapter}{##2第##1章##3}%
  \crefname{section}{第}{第}%
  \crefformat{section}{##2第##1節##3}%
  \crefname{subsection}{第}{第}%
  \crefformat{subsection}{##2第##1項##3}%
  \crefname{figure}{図}{図}%
  \crefformat{figure}{##2図##1##3}%
  \crefname{table}{表}{表}%
  \crefformat{table}{##2表##1##3}%
  \crefname{appendix}{付録}{付録}%
  \crefformat{appendix}{##2付録##1##3}%
  \crefname{equation}{式}{式}%
  \crefformat{equation}{##2式##1##3}%
  \cref{#1}%
  \endgroup
}
% ****************************************************************************************************


% ****************************************************************************************************
% 3. Loading some handy packages
% ****************************************************************************************************
% ********************************************************************
% Packages with options that might require adjustments
% ********************************************************************
\PassOptionsToPackage{ngerman,american}{babel} % change this to your language(s), main language last
% Spanish languages need extra options in order to work with this template
%\PassOptionsToPackage{spanish,es-lcroman}{babel}
    \usepackage{babel}

\usepackage{csquotes}
\PassOptionsToPackage{%
  backend=biber,%
  bibencoding=utf8,%
  language=auto,%
  style=apa,%
  sorting=nyt,%
  maxbibnames=20,%
  giveninits=false%
}{biblatex}
    \usepackage{biblatex}
    \setlength{\bibhang}{2em} % Hanging indentation of 2 Japanese characters
    
    % Japanese Paragraph Indentation
    \usepackage{indentfirst} % Indent first paragraph after section header
    \AtBeginDocument{\setlength{\parindent}{1em}} % Indent width: 1 full-width character
    
    % Fix for bibliography wrapping issues with justification
    \appto{\bibsetup}{\setlength{\emergencystretch}{3em}}

\DeclareLanguageMapping{american}{american-apa}
\DeclareLanguageMapping{english}{english-apa}
\DeclareLanguageMapping{japanese}{english-apa} % Use English APA as base for Japanese


% ********************************************************************
% Custom sorting: use yomi field for Japanese entries
% ********************************************************************
% Map yomi to sortname for Japanese entries only
% Biber accepts custom fields in sourcemaps without declaration
\DeclareSourcemap{
  \maps[datatype=bibtex]{
    \map[overwrite]{
      \step[fieldsource=langid, match=japanese, final]
      \step[fieldsource=yomi, fieldset=sortname, origfieldval]
    }
    \map[overwrite]{
      \step[typesource=incollectionja, typetarget=incollection, final]
      \step[fieldset=keywords, fieldvalue=incollectionja, append]
      \step[fieldset=langid, fieldvalue=japanese]
    }
  }
}

% ********************************************************************
% Japanese bibliography formatting
% ********************************************************************
% For Japanese entries, use:
%   author = {細田, 衛士 and 山本, 雅資}
%   langid = {japanese}
%   yomi = {ほそだ, えいじ and やまもと, まさし}
% Result: 細田衛士・山本雅資 (full name, no comma, ・ separator)
% Sorting: Uses hiragana yomi field for alphabetical ordering

% Change fixed labels for Japanese references
% Change fixed labels for Japanese references
\DefineBibliographyStrings{japanese}{
  in      = {},   % suppress the leading “In”
  editor  = {編}, % editor -> 編
  editors = {編}, % editors -> 編
}

% Override "In" macro
\renewbibmacro*{in}{%
  \ifboolexpr{
    test {\iffieldequalstr{langid}{japanese}}
    or
    test {\ifkeyword{incollectionja}}
  }
    {}
    {\printtext{\bibstring{in}\intitlepunct}}%
}

% Override editor string macro
\renewbibmacro*{apaeditorstrg}[1]{%
  \ifboolexpr{
    test {\iffieldequalstr{langid}{japanese}}
    or
    test {\ifkeyword{incollectionja}}
  }
    {編}
    {\iffieldundef{#1type}
       {\ifthenelse{\value{#1}>1\OR\ifandothers{#1}}
          {\bibcpstring{editors}}
          {\bibcpstring{editor}}}
       {\ifthenelse{\value{#1}>1\OR\ifandothers{#1}}
            {\bibcpstring{type\thefield{#1type}s}}
            {\bibcpstring{type\thefield{#1type}}}}}%
}

% Override the low-level name format used by APA
\DeclareNameFormat{apaauthor}{%
  \iffieldequalstr{langid}{japanese}
    {% Japanese: Family + Given, no comma, no abbreviation
      \namepartfamily
      \namepartgiven
      % Add ・ delimiter between names, but not after the last one
      \ifnumequal{\value{listcount}}{\value{liststop}}
        {}
        {・}%
    }
    {% Non-Japanese: standard APA format
      \ifcase\value{uniquename}%
        \usebibmacro{name:family-given}
          {\namepartfamily}
          {\namepartgiveni}
          {\namepartprefix}
          {\namepartsuffix}%
      \or
        \ifuseprefix
          {\usebibmacro{name:given-family}
            {\namepartfamily}
            {\namepartgiveni}
            {\namepartprefix}
            {\namepartsuffix}}
          {\usebibmacro{name:given-family}
            {\namepartfamily}
            {\namepartgiveni}
            {\namepartprefix}
            {\namepartsuffix}}%
      \or
        \usebibmacro{name:family-given}
          {\namepartfamily}
          {\namepartgiveni}
          {\namepartprefix}
          {\namepartsuffix}%
      \fi
    }%
  \usebibmacro{name:andothers}%
}

% Use the same format for editors
\DeclareNameAlias{apaeditor}{apaauthor}
\DeclareNameAlias{apatranslator}{apaauthor}
\DeclareNameAlias{apanames}{apaauthor}

% Also apply to citations (labelname is used for \cite, \parencite, etc.)
\DeclareNameFormat{labelname}{%
  \iffieldequalstr{langid}{japanese}
    {% Japanese: Family only for citations, with ・ delimiter
      \namepartfamily
      \ifnumequal{\value{listcount}}{\value{liststop}}
        {}
        {・}%
    }
    {% Non-Japanese: standard APA citation format
      \ifcase\value{uniquename}%
        \usebibmacro{name:family}
          {\namepartfamily}
          {\namepartgiven}
          {\namepartprefix}
          {\namepartsuffix}%
      \or
        \ifuseprefix
          {\usebibmacro{name:given-family}
            {\namepartfamily}
            {\namepartgiveni}
            {\namepartprefix}
            {\namepartsuffixi}}
          {\usebibmacro{name:given-family}
            {\namepartfamily}
            {\namepartgiveni}
            {\namepartprefix}
            {\namepartsuffixi}}%
      \or
        \usebibmacro{name:given-family}
          {\namepartfamily}
          {\namepartgiven}
          {\namepartprefix}
          {\namepartsuffix}%
      \fi
    }%
  \usebibmacro{name:andothers}%
}

% Override delimiters for Japanese entries globally
\DeclareDelimFormat{multinamedelim}{%
  \iffieldequalstr{langid}{japanese}
    {・}%
    {\addcomma\space}%
}

\DeclareDelimFormat{finalnamedelim}{%
  \iffieldequalstr{langid}{japanese}
    {・}%
    {\space\&\space}%
}

% Override "et al." to show "他" for Japanese entries
\renewbibmacro*{name:andothers}{%
  \ifboolexpr{
    test {\ifnumequal{\value{listcount}}{\value{liststop}}}
    and
    test \ifmorenames
  }
    {%
      \iffieldequalstr{langid}{japanese}
        {他}% Japanese: use 他 instead of et al.
        {\printdelim{andothersdelim}\bibstring{andothers}}% Non-Japanese: use et al.
    }
    {}%
}

% Disable italics for Japanese titles
\DeclareFieldFormat[article,book,inbook,incollection,inproceedings,report,thesis,unpublished]{title}{%
  \iffieldequalstr{langid}{japanese}
    {#1}% Japanese: no italics
    {\mkbibemph{#1}}% Non-Japanese: use italics
}

\DeclareFieldFormat[article,book,inbook,incollection,inproceedings,report,thesis,unpublished]{journaltitle}{%
  \iffieldequalstr{langid}{japanese}
    {#1}% Japanese: no italics
    {\mkbibemph{#1}}% Non-Japanese: use italics
}

\DeclareFieldFormat[article,book,inbook,incollection,inproceedings,report,thesis,unpublished]{booktitle}{%
  \iffieldequalstr{langid}{japanese}
    {#1}% Japanese: no italics
    {\mkbibemph{#1}}% Non-Japanese: use italics
}


\PassOptionsToPackage{fleqn}{amsmath}       % math environments and more by the AMS
  \usepackage{amsmath}

% ********************************************************************
% General useful packages
% ********************************************************************
\usepackage{graphicx} %
\usepackage{scrhack} % fix warnings when using KOMA with listings package
\usepackage{xspace} % to get the spacing after macros right
\PassOptionsToPackage{printonlyused,smaller}{acronym}
  \usepackage{acronym} % nice macros for handling all acronyms in the thesis
  %\renewcommand{\bflabel}[1]{{#1}\hfill} % fix the list of acronyms --> no longer working
  %\renewcommand*{\acsfont}[1]{\textsc{#1}}
  %\renewcommand*{\aclabelfont}[1]{\acsfont{#1}}
  %\def\bflabel#1{{#1\hfill}}
  \def\bflabel#1{{\acsfont{#1}\hfill}}
  \def\aclabelfont#1{\acsfont{#1}}
% ****************************************************************************************************
%\usepackage{pgfplots} % External TikZ/PGF support (thanks to Andreas Nautsch)
%\usetikzlibrary{external}
%\tikzexternalize[mode=list and make, prefix=ext-tikz/]
% ****************************************************************************************************


% ****************************************************************************************************
% 4. Setup floats: tables, (sub)figures, and captions
% ****************************************************************************************************
\usepackage{tabularx} % better tables
  \setlength{\extrarowheight}{3pt} % increase table row height
\newcommand{\tableheadline}[1]{\multicolumn{1}{l}{\spacedlowsmallcaps{#1}}}
\newcommand{\myfloatalign}{\centering} % to be used with each float for alignment
\usepackage{subfig}
% ****************************************************************************************************


% ****************************************************************************************************
% 5. Setup code listings
% ****************************************************************************************************
\usepackage{listings}
%\lstset{emph={trueIndex,root},emphstyle=\color{BlueViolet}}%\underbar} % for special keywords
\lstset{language=[LaTeX]Tex,%C++,
  morekeywords={PassOptionsToPackage,selectlanguage},
  keywordstyle=\color{RoyalBlue},%\bfseries,
  basicstyle=\small\ttfamily,
  %identifierstyle=\color{NavyBlue},
  commentstyle=\color{Green}\ttfamily,
  stringstyle=\rmfamily,
  numbers=none,%left,%
  numberstyle=\scriptsize,%\tiny
  stepnumber=5,
  numbersep=8pt,
  showstringspaces=false,
  breaklines=true,
  %frameround=ftff,
  %frame=single,
  belowcaptionskip=.75\baselineskip
  %frame=L
}
% ****************************************************************************************************


% ****************************************************************************************************
% 6. Last calls before the bar closes
% ****************************************************************************************************
% ********************************************************************
% Her Majesty herself
% ********************************************************************
\PassOptionsToPackage{hyperfootnotes=false}{hyperref}
\usepackage{classicthesis}


% ********************************************************************
% Fine-tune hyperreferences (hyperref should be called last)
% ********************************************************************
\hypersetup{%
  %draft, % hyperref's draft mode, for printing see below
  colorlinks=true, linktocpage=true, pdfstartpage=3, pdfstartview=FitV,%
  % uncomment the following line if you want to have black links (e.g., for printing)
  %colorlinks=false, linktocpage=false, pdfstartpage=3, pdfstartview=FitV, pdfborder={0 0 0},%
  breaklinks=true, pageanchor=true,%
  pdfpagemode=UseNone, %
  % pdfpagemode=UseOutlines,%
  plainpages=false, bookmarksnumbered, bookmarksopen=true, bookmarksopenlevel=1,%
  hypertexnames=true, pdfhighlight=/O,%nesting=true,%frenchlinks,%
  urlcolor=CTurl, linkcolor=CTlink, citecolor=CTcitation, %pagecolor=RoyalBlue,%
  %urlcolor=Black, linkcolor=Black, citecolor=Black, %pagecolor=Black,%
  pdftitle={\myTitle},%
  pdfauthor={\textcopyright\ \myName, \myUni, \myFaculty},%
  pdfsubject={},%
  pdfkeywords={},%
  pdfcreator={pdfLaTeX},%
  pdfproducer={LaTeX with hyperref and classicthesis}%
}
% ********************************************************************
% Setup autoreferences (hyperref and babel)
% ********************************************************************
% There are some issues regarding autorefnames
% http://www.tex.ac.uk/cgi-bin/texfaq2html?label=latexwords
% you have to redefine the macros for the
% language you use, e.g., american, ngerman
% (as chosen when loading babel/AtBeginDocument)
% ********************************************************************
 \makeatletter
 \@ifpackageloaded{babel}%
   {%
     \addto\extrasamerican{%
       \renewcommand*{\figureautorefname}{Figure}%
       \renewcommand*{\tableautorefname}{Table}%
       \renewcommand*{\partautorefname}{Part}%
       \renewcommand*{\chapterautorefname}{Chapter}%
       \renewcommand*{\sectionautorefname}{Section}%
       \renewcommand*{\subsectionautorefname}{Section}%
       \renewcommand*{\subsubsectionautorefname}{Section}%
       \renewcommand*{\figurename}{図}%
       \renewcommand*{\tablename}{表}%
     }%
     \addto\extrasngerman{%
       \renewcommand*{\paragraphautorefname}{Absatz}%
       \renewcommand*{\subparagraphautorefname}{Unterabsatz}%
       \renewcommand*{\footnoteautorefname}{Fu\"snote}%
       \renewcommand*{\FancyVerbLineautorefname}{Zeile}%
       \renewcommand*{\theoremautorefname}{Theorem}%
       \renewcommand*{\appendixautorefname}{Anhang}%
       \renewcommand*{\equationautorefname}{Gleichung}%
       \renewcommand*{\itemautorefname}{Punkt}%
     }%
       % Fix to getting autorefs for subfigures right (thanks to Belinda Vogt for changing the definition)
       \providecommand{\subfigureautorefname}{\figureautorefname}%
     }{\relax}
 \makeatother

% (Better) alternative to \autoref is \cref via the cleveref package
\usepackage{cleveref}
%\crefformat{part}{Part #2\MakeUppercase{#1}#3}


% ********************************************************************
% Development Stuff
% ********************************************************************
\listfiles
%\PassOptionsToPackage{l2tabu,orthodox,abort}{nag}
%  \usepackage{nag}
%\PassOptionsToPackage{warning, all}{onlyamsmath}
%  \usepackage{onlyamsmath}


% ****************************************************************************************************
% 7. Further adjustments (experimental)
% ****************************************************************************************************
% ********************************************************************
% Changing the text area
% ********************************************************************
\areaset[current]{370pt}{761pt} % Increased width from default 336pt
%\setlength{\marginparwidth}{7em}%
%\setlength{\marginparsep}{2em}%

% ********************************************************************
% Using different fonts
% ********************************************************************
%\usepackage[oldstylenums]{kpfonts} % oldstyle notextcomp
% \usepackage[osf]{libertine}
%\usepackage[light,condensed,math]{iwona}
%\renewcommand{\sfdefault}{iwona}
%\usepackage{lmodern} % <-- no osf support :-(
%\usepackage{cfr-lm} %
%\usepackage[urw-garamond]{mathdesign} <-- no osf support :-(
%\usepackage[default,osfigures]{opensans} % scale=0.95
%\usepackage[sfdefault]{FiraSans}
%\usepackage[opticals,mathlf]{MinionPro} % onlytext
% ********************************************************************
%\setmainfont[Ligatures=TeX]{Times New Roman} % Set Times New Roman as main font
% ********************************************************************
%\usepackage[largesc,osf]{newpxtext}
%\linespread{1.05} % a bit more for Palatino
% Used to fix these:
% https://bitbucket.org/amiede/classicthesis/issues/139/italics-in-pallatino-capitals-chapter
% https://bitbucket.org/amiede/classicthesis/issues/45/problema-testatine-su-classicthesis-style
% ********************************************************************
% ****************************************************************************************************

% ****************************************************************************************************
% If you like the classicthesis, then I would appreciate a postcard.
% My address can be found in the file ClassicThesis.pdf. A collection
% of the postcards I received so far is available online at
% http://postcards.miede.de
% ****************************************************************************************************


% ****************************************************************************************************
% 0. Set the encoding of your files. UTF-8 is the only sensible encoding nowadays. If you can't read
% äöüßáéçèê∂åëæƒÏ€ then change the encoding setting in your editor, not the line below. If your editor
% does not support utf8 use another editor!
% ****************************************************************************************************
\usepackage{iftex}
\ifPDFTeX
  \PassOptionsToPackage{utf8}{inputenc}
  \usepackage{inputenc}
  \PassOptionsToPackage{T1}{fontenc} % T2A for cyrillics
  \usepackage{fontenc}
\else
  \usepackage{fontspec}
\fi


% ****************************************************************************************************
% 1. Configure classicthesis for your needs here, e.g., remove "drafting" below
% in order to deactivate the time-stamp on the pages
% (see ClassicThesis.pdf for more information):
% ****************************************************************************************************
\PassOptionsToPackage{
  drafting=false,    % print version information on the bottom of the pages
  tocaligned=false, % the left column of the toc will be aligned (no indentation)
  dottedtoc=false,  % page numbers in ToC flushed right
  eulerchapternumbers=false, % use AMS Euler for chapter font (otherwise Palatino)
  floatperchapter=true,     % numbering per chapter for all floats (i.e., Figure 1.1)
  eulermath=false,  % use awesome Euler fonts for mathematical formulae (only with pdfLaTeX)
  beramono=true,    % toggle a nice monospaced font (w/ bold)
  palatino=false,    % deactivate standard font for loading another one, see the last section at the end of this file for suggestions
  %linedheaders=true, % obsolete / available for backwards compatibility
  style=classicthesis % classicthesis, arsclassica, linedheaders, plain
}{classicthesis}

% ****************************************************************************************************
% 2. Personal data and user ad-hoc commands (insert your own data here)
% ****************************************************************************************************
% ****************************************************************************************************
% 2. Personal data and user ad-hoc commands (insert your own data here)
% ****************************************************************************************************
\newcommand{\myTitle}{The Title of the Thesis\xspace}
\newcommand{\mySubtitle}{The Subtitle of the Thesis\xspace}
\newcommand{\myDegree}{Your Degree\xspace}
\newcommand{\myName}{Your Name\xspace}
\newcommand{\myProf}{Your Supervisor\xspace}
\newcommand{\myOtherProf}{Your Other Supervisor\xspace}
\newcommand{\mySupervisor}{Your Supervisor\xspace}
\newcommand{\myFaculty}{Your Faculty\xspace}
\newcommand{\myDepartment}{Your Department\xspace}
\newcommand{\myUni}{Your Uni\xspace}
\newcommand{\myLocation}{Location\xspace}
\newcommand{\myTime}{Month Year\xspace}
\newcommand{\myVersion}{v1.0\xspace}

% ********************************************************************
% Setup, finetuning, and useful commands
% ********************************************************************
\providecommand{\mLyX}{L\kern-.1667em\lower.25em\hbox{Y}\kern-.125emX\@}
\newcommand{\ie}{i.\,e.}
\newcommand{\Ie}{I.\,e.}
\newcommand{\eg}{e.\,g.}
\newcommand{\Eg}{E.\,g.}
\newcommand{\autorefja}[1]{%
  \begingroup
  \crefname{chapter}{第}{第}%
  \crefformat{chapter}{##2第##1章##3}%
  \crefname{section}{第}{第}%
  \crefformat{section}{##2第##1節##3}%
  \crefname{subsection}{第}{第}%
  \crefformat{subsection}{##2第##1項##3}%
  \crefname{figure}{図}{図}%
  \crefformat{figure}{##2図##1##3}%
  \crefname{table}{表}{表}%
  \crefformat{table}{##2表##1##3}%
  \crefname{appendix}{付録}{付録}%
  \crefformat{appendix}{##2付録##1##3}%
  \crefname{equation}{式}{式}%
  \crefformat{equation}{##2式##1##3}%
  \cref{#1}%
  \endgroup
}
% ****************************************************************************************************


% ****************************************************************************************************
% 3. Loading some handy packages
% ****************************************************************************************************
% ********************************************************************
% Packages with options that might require adjustments
% ********************************************************************
\PassOptionsToPackage{ngerman,american}{babel} % change this to your language(s), main language last
% Spanish languages need extra options in order to work with this template
%\PassOptionsToPackage{spanish,es-lcroman}{babel}
    \usepackage{babel}

\usepackage{csquotes}
\PassOptionsToPackage{%
  backend=biber,%
  bibencoding=utf8,%
  language=auto,%
  style=apa,%
  sorting=nyt,%
  maxbibnames=20,%
  giveninits=false%
}{biblatex}
    \usepackage{biblatex}
    \setlength{\bibhang}{2em} % Hanging indentation of 2 Japanese characters
    
    % Japanese Paragraph Indentation
    \usepackage{indentfirst} % Indent first paragraph after section header
    \AtBeginDocument{\setlength{\parindent}{1em}} % Indent width: 1 full-width character
    
    % Fix for bibliography wrapping issues with justification
    \appto{\bibsetup}{\setlength{\emergencystretch}{3em}}

\DeclareLanguageMapping{american}{american-apa}
\DeclareLanguageMapping{english}{english-apa}
\DeclareLanguageMapping{japanese}{english-apa} % Use English APA as base for Japanese


% ********************************************************************
% Custom sorting: use yomi field for Japanese entries
% ********************************************************************
% Map yomi to sortname for Japanese entries only
% Biber accepts custom fields in sourcemaps without declaration
\DeclareSourcemap{
  \maps[datatype=bibtex]{
    \map[overwrite]{
      \step[fieldsource=langid, match=japanese, final]
      \step[fieldsource=yomi, fieldset=sortname, origfieldval]
    }
    \map[overwrite]{
      \step[typesource=incollectionja, typetarget=incollection, final]
      \step[fieldset=keywords, fieldvalue=incollectionja, append]
      \step[fieldset=langid, fieldvalue=japanese]
    }
  }
}

% ********************************************************************
% Japanese bibliography formatting
% ********************************************************************
% For Japanese entries, use:
%   author = {細田, 衛士 and 山本, 雅資}
%   langid = {japanese}
%   yomi = {ほそだ, えいじ and やまもと, まさし}
% Result: 細田衛士・山本雅資 (full name, no comma, ・ separator)
% Sorting: Uses hiragana yomi field for alphabetical ordering

% Change fixed labels for Japanese references
% Change fixed labels for Japanese references
\DefineBibliographyStrings{japanese}{
  in      = {},   % suppress the leading “In”
  editor  = {編}, % editor -> 編
  editors = {編}, % editors -> 編
}

% Override "In" macro
\renewbibmacro*{in}{%
  \ifboolexpr{
    test {\iffieldequalstr{langid}{japanese}}
    or
    test {\ifkeyword{incollectionja}}
  }
    {}
    {\printtext{\bibstring{in}\intitlepunct}}%
}

% Override editor string macro
\renewbibmacro*{apaeditorstrg}[1]{%
  \ifboolexpr{
    test {\iffieldequalstr{langid}{japanese}}
    or
    test {\ifkeyword{incollectionja}}
  }
    {編}
    {\iffieldundef{#1type}
       {\ifthenelse{\value{#1}>1\OR\ifandothers{#1}}
          {\bibcpstring{editors}}
          {\bibcpstring{editor}}}
       {\ifthenelse{\value{#1}>1\OR\ifandothers{#1}}
            {\bibcpstring{type\thefield{#1type}s}}
            {\bibcpstring{type\thefield{#1type}}}}}%
}

% Override the low-level name format used by APA
\DeclareNameFormat{apaauthor}{%
  \iffieldequalstr{langid}{japanese}
    {% Japanese: Family + Given, no comma, no abbreviation
      \namepartfamily
      \namepartgiven
      % Add ・ delimiter between names, but not after the last one
      \ifnumequal{\value{listcount}}{\value{liststop}}
        {}
        {・}%
    }
    {% Non-Japanese: standard APA format
      \ifcase\value{uniquename}%
        \usebibmacro{name:family-given}
          {\namepartfamily}
          {\namepartgiveni}
          {\namepartprefix}
          {\namepartsuffix}%
      \or
        \ifuseprefix
          {\usebibmacro{name:given-family}
            {\namepartfamily}
            {\namepartgiveni}
            {\namepartprefix}
            {\namepartsuffix}}
          {\usebibmacro{name:given-family}
            {\namepartfamily}
            {\namepartgiveni}
            {\namepartprefix}
            {\namepartsuffix}}%
      \or
        \usebibmacro{name:family-given}
          {\namepartfamily}
          {\namepartgiveni}
          {\namepartprefix}
          {\namepartsuffix}%
      \fi
    }%
  \usebibmacro{name:andothers}%
}

% Use the same format for editors
\DeclareNameAlias{apaeditor}{apaauthor}
\DeclareNameAlias{apatranslator}{apaauthor}
\DeclareNameAlias{apanames}{apaauthor}

% Also apply to citations (labelname is used for \cite, \parencite, etc.)
\DeclareNameFormat{labelname}{%
  \iffieldequalstr{langid}{japanese}
    {% Japanese: Family only for citations, with ・ delimiter
      \namepartfamily
      \ifnumequal{\value{listcount}}{\value{liststop}}
        {}
        {・}%
    }
    {% Non-Japanese: standard APA citation format
      \ifcase\value{uniquename}%
        \usebibmacro{name:family}
          {\namepartfamily}
          {\namepartgiven}
          {\namepartprefix}
          {\namepartsuffix}%
      \or
        \ifuseprefix
          {\usebibmacro{name:given-family}
            {\namepartfamily}
            {\namepartgiveni}
            {\namepartprefix}
            {\namepartsuffixi}}
          {\usebibmacro{name:given-family}
            {\namepartfamily}
            {\namepartgiveni}
            {\namepartprefix}
            {\namepartsuffixi}}%
      \or
        \usebibmacro{name:given-family}
          {\namepartfamily}
          {\namepartgiven}
          {\namepartprefix}
          {\namepartsuffix}%
      \fi
    }%
  \usebibmacro{name:andothers}%
}

% Override delimiters for Japanese entries globally
\DeclareDelimFormat{multinamedelim}{%
  \iffieldequalstr{langid}{japanese}
    {・}%
    {\addcomma\space}%
}

\DeclareDelimFormat{finalnamedelim}{%
  \iffieldequalstr{langid}{japanese}
    {・}%
    {\space\&\space}%
}

% Override "et al." to show "他" for Japanese entries
\renewbibmacro*{name:andothers}{%
  \ifboolexpr{
    test {\ifnumequal{\value{listcount}}{\value{liststop}}}
    and
    test \ifmorenames
  }
    {%
      \iffieldequalstr{langid}{japanese}
        {他}% Japanese: use 他 instead of et al.
        {\printdelim{andothersdelim}\bibstring{andothers}}% Non-Japanese: use et al.
    }
    {}%
}

% Disable italics for Japanese titles
\DeclareFieldFormat[article,book,inbook,incollection,inproceedings,report,thesis,unpublished]{title}{%
  \iffieldequalstr{langid}{japanese}
    {#1}% Japanese: no italics
    {\mkbibemph{#1}}% Non-Japanese: use italics
}

\DeclareFieldFormat[article,book,inbook,incollection,inproceedings,report,thesis,unpublished]{journaltitle}{%
  \iffieldequalstr{langid}{japanese}
    {#1}% Japanese: no italics
    {\mkbibemph{#1}}% Non-Japanese: use italics
}

\DeclareFieldFormat[article,book,inbook,incollection,inproceedings,report,thesis,unpublished]{booktitle}{%
  \iffieldequalstr{langid}{japanese}
    {#1}% Japanese: no italics
    {\mkbibemph{#1}}% Non-Japanese: use italics
}


\PassOptionsToPackage{fleqn}{amsmath}       % math environments and more by the AMS
  \usepackage{amsmath}

% ********************************************************************
% General useful packages
% ********************************************************************
\usepackage{graphicx} %
\usepackage{scrhack} % fix warnings when using KOMA with listings package
\usepackage{xspace} % to get the spacing after macros right
\PassOptionsToPackage{printonlyused,smaller}{acronym}
  \usepackage{acronym} % nice macros for handling all acronyms in the thesis
  %\renewcommand{\bflabel}[1]{{#1}\hfill} % fix the list of acronyms --> no longer working
  %\renewcommand*{\acsfont}[1]{\textsc{#1}}
  %\renewcommand*{\aclabelfont}[1]{\acsfont{#1}}
  %\def\bflabel#1{{#1\hfill}}
  \def\bflabel#1{{\acsfont{#1}\hfill}}
  \def\aclabelfont#1{\acsfont{#1}}
% ****************************************************************************************************
%\usepackage{pgfplots} % External TikZ/PGF support (thanks to Andreas Nautsch)
%\usetikzlibrary{external}
%\tikzexternalize[mode=list and make, prefix=ext-tikz/]
% ****************************************************************************************************


% ****************************************************************************************************
% 4. Setup floats: tables, (sub)figures, and captions
% ****************************************************************************************************
\usepackage{tabularx} % better tables
  \setlength{\extrarowheight}{3pt} % increase table row height
\newcommand{\tableheadline}[1]{\multicolumn{1}{l}{\spacedlowsmallcaps{#1}}}
\newcommand{\myfloatalign}{\centering} % to be used with each float for alignment
\usepackage{subfig}
% ****************************************************************************************************


% ****************************************************************************************************
% 5. Setup code listings
% ****************************************************************************************************
\usepackage{listings}
%\lstset{emph={trueIndex,root},emphstyle=\color{BlueViolet}}%\underbar} % for special keywords
\lstset{language=[LaTeX]Tex,%C++,
  morekeywords={PassOptionsToPackage,selectlanguage},
  keywordstyle=\color{RoyalBlue},%\bfseries,
  basicstyle=\small\ttfamily,
  %identifierstyle=\color{NavyBlue},
  commentstyle=\color{Green}\ttfamily,
  stringstyle=\rmfamily,
  numbers=none,%left,%
  numberstyle=\scriptsize,%\tiny
  stepnumber=5,
  numbersep=8pt,
  showstringspaces=false,
  breaklines=true,
  %frameround=ftff,
  %frame=single,
  belowcaptionskip=.75\baselineskip
  %frame=L
}
% ****************************************************************************************************


% ****************************************************************************************************
% 6. Last calls before the bar closes
% ****************************************************************************************************
% ********************************************************************
% Her Majesty herself
% ********************************************************************
\PassOptionsToPackage{hyperfootnotes=false}{hyperref}
\usepackage{classicthesis}


% ********************************************************************
% Fine-tune hyperreferences (hyperref should be called last)
% ********************************************************************
\hypersetup{%
  %draft, % hyperref's draft mode, for printing see below
  colorlinks=true, linktocpage=true, pdfstartpage=3, pdfstartview=FitV,%
  % uncomment the following line if you want to have black links (e.g., for printing)
  %colorlinks=false, linktocpage=false, pdfstartpage=3, pdfstartview=FitV, pdfborder={0 0 0},%
  breaklinks=true, pageanchor=true,%
  pdfpagemode=UseNone, %
  % pdfpagemode=UseOutlines,%
  plainpages=false, bookmarksnumbered, bookmarksopen=true, bookmarksopenlevel=1,%
  hypertexnames=true, pdfhighlight=/O,%nesting=true,%frenchlinks,%
  urlcolor=CTurl, linkcolor=CTlink, citecolor=CTcitation, %pagecolor=RoyalBlue,%
  %urlcolor=Black, linkcolor=Black, citecolor=Black, %pagecolor=Black,%
  pdftitle={\myTitle},%
  pdfauthor={\textcopyright\ \myName, \myUni, \myFaculty},%
  pdfsubject={},%
  pdfkeywords={},%
  pdfcreator={pdfLaTeX},%
  pdfproducer={LaTeX with hyperref and classicthesis}%
}
% ********************************************************************
% Setup autoreferences (hyperref and babel)
% ********************************************************************
% There are some issues regarding autorefnames
% http://www.tex.ac.uk/cgi-bin/texfaq2html?label=latexwords
% you have to redefine the macros for the
% language you use, e.g., american, ngerman
% (as chosen when loading babel/AtBeginDocument)
% ********************************************************************
 \makeatletter
 \@ifpackageloaded{babel}%
   {%
     \addto\extrasamerican{%
       \renewcommand*{\figureautorefname}{Figure}%
       \renewcommand*{\tableautorefname}{Table}%
       \renewcommand*{\partautorefname}{Part}%
       \renewcommand*{\chapterautorefname}{Chapter}%
       \renewcommand*{\sectionautorefname}{Section}%
       \renewcommand*{\subsectionautorefname}{Section}%
       \renewcommand*{\subsubsectionautorefname}{Section}%
       \renewcommand*{\figurename}{図}%
       \renewcommand*{\tablename}{表}%
     }%
     \addto\extrasngerman{%
       \renewcommand*{\paragraphautorefname}{Absatz}%
       \renewcommand*{\subparagraphautorefname}{Unterabsatz}%
       \renewcommand*{\footnoteautorefname}{Fu\"snote}%
       \renewcommand*{\FancyVerbLineautorefname}{Zeile}%
       \renewcommand*{\theoremautorefname}{Theorem}%
       \renewcommand*{\appendixautorefname}{Anhang}%
       \renewcommand*{\equationautorefname}{Gleichung}%
       \renewcommand*{\itemautorefname}{Punkt}%
     }%
       % Fix to getting autorefs for subfigures right (thanks to Belinda Vogt for changing the definition)
       \providecommand{\subfigureautorefname}{\figureautorefname}%
     }{\relax}
 \makeatother

% (Better) alternative to \autoref is \cref via the cleveref package
\usepackage{cleveref}
%\crefformat{part}{Part #2\MakeUppercase{#1}#3}


% ********************************************************************
% Development Stuff
% ********************************************************************
\listfiles
%\PassOptionsToPackage{l2tabu,orthodox,abort}{nag}
%  \usepackage{nag}
%\PassOptionsToPackage{warning, all}{onlyamsmath}
%  \usepackage{onlyamsmath}


% ****************************************************************************************************
% 7. Further adjustments (experimental)
% ****************************************************************************************************
% ********************************************************************
% Changing the text area
% ********************************************************************
\areaset[current]{370pt}{761pt} % Increased width from default 336pt
%\setlength{\marginparwidth}{7em}%
%\setlength{\marginparsep}{2em}%

% ********************************************************************
% Using different fonts
% ********************************************************************
%\usepackage[oldstylenums]{kpfonts} % oldstyle notextcomp
% \usepackage[osf]{libertine}
%\usepackage[light,condensed,math]{iwona}
%\renewcommand{\sfdefault}{iwona}
%\usepackage{lmodern} % <-- no osf support :-(
%\usepackage{cfr-lm} %
%\usepackage[urw-garamond]{mathdesign} <-- no osf support :-(
%\usepackage[default,osfigures]{opensans} % scale=0.95
%\usepackage[sfdefault]{FiraSans}
%\usepackage[opticals,mathlf]{MinionPro} % onlytext
% ********************************************************************
%\setmainfont[Ligatures=TeX]{Times New Roman} % Set Times New Roman as main font
% ********************************************************************
%\usepackage[largesc,osf]{newpxtext}
%\linespread{1.05} % a bit more for Palatino
% Used to fix these:
% https://bitbucket.org/amiede/classicthesis/issues/139/italics-in-pallatino-capitals-chapter
% https://bitbucket.org/amiede/classicthesis/issues/45/problema-testatine-su-classicthesis-style
% ********************************************************************
% ****************************************************************************************************

% ****************************************************************************************************
% If you like the classicthesis, then I would appreciate a postcard.
% My address can be found in the file ClassicThesis.pdf. A collection
% of the postcards I received so far is available online at
% http://postcards.miede.de
% ****************************************************************************************************


% ****************************************************************************************************
% 0. Set the encoding of your files. UTF-8 is the only sensible encoding nowadays. If you can't read
% äöüßáéçèê∂åëæƒÏ€ then change the encoding setting in your editor, not the line below. If your editor
% does not support utf8 use another editor!
% ****************************************************************************************************
\usepackage{iftex}
\ifPDFTeX
  \PassOptionsToPackage{utf8}{inputenc}
  \usepackage{inputenc}
  \PassOptionsToPackage{T1}{fontenc} % T2A for cyrillics
  \usepackage{fontenc}
\else
  \usepackage{fontspec}
\fi


% ****************************************************************************************************
% 1. Configure classicthesis for your needs here, e.g., remove "drafting" below
% in order to deactivate the time-stamp on the pages
% (see ClassicThesis.pdf for more information):
% ****************************************************************************************************
\PassOptionsToPackage{
  drafting=false,    % print version information on the bottom of the pages
  tocaligned=false, % the left column of the toc will be aligned (no indentation)
  dottedtoc=false,  % page numbers in ToC flushed right
  eulerchapternumbers=false, % use AMS Euler for chapter font (otherwise Palatino)
  floatperchapter=true,     % numbering per chapter for all floats (i.e., Figure 1.1)
  eulermath=false,  % use awesome Euler fonts for mathematical formulae (only with pdfLaTeX)
  beramono=true,    % toggle a nice monospaced font (w/ bold)
  palatino=false,    % deactivate standard font for loading another one, see the last section at the end of this file for suggestions
  %linedheaders=true, % obsolete / available for backwards compatibility
  style=classicthesis % classicthesis, arsclassica, linedheaders, plain
}{classicthesis}

% ****************************************************************************************************
% 2. Personal data and user ad-hoc commands (insert your own data here)
% ****************************************************************************************************
% ****************************************************************************************************
% 2. Personal data and user ad-hoc commands (insert your own data here)
% ****************************************************************************************************
\newcommand{\myTitle}{The Title of the Thesis\xspace}
\newcommand{\mySubtitle}{The Subtitle of the Thesis\xspace}
\newcommand{\myDegree}{Your Degree\xspace}
\newcommand{\myName}{Your Name\xspace}
\newcommand{\myProf}{Your Supervisor\xspace}
\newcommand{\myOtherProf}{Your Other Supervisor\xspace}
\newcommand{\mySupervisor}{Your Supervisor\xspace}
\newcommand{\myFaculty}{Your Faculty\xspace}
\newcommand{\myDepartment}{Your Department\xspace}
\newcommand{\myUni}{Your Uni\xspace}
\newcommand{\myLocation}{Location\xspace}
\newcommand{\myTime}{Month Year\xspace}
\newcommand{\myVersion}{v1.0\xspace}

% ********************************************************************
% Setup, finetuning, and useful commands
% ********************************************************************
\providecommand{\mLyX}{L\kern-.1667em\lower.25em\hbox{Y}\kern-.125emX\@}
\newcommand{\ie}{i.\,e.}
\newcommand{\Ie}{I.\,e.}
\newcommand{\eg}{e.\,g.}
\newcommand{\Eg}{E.\,g.}
\newcommand{\autorefja}[1]{%
  \begingroup
  \crefname{chapter}{第}{第}%
  \crefformat{chapter}{##2第##1章##3}%
  \crefname{section}{第}{第}%
  \crefformat{section}{##2第##1節##3}%
  \crefname{subsection}{第}{第}%
  \crefformat{subsection}{##2第##1項##3}%
  \crefname{figure}{図}{図}%
  \crefformat{figure}{##2図##1##3}%
  \crefname{table}{表}{表}%
  \crefformat{table}{##2表##1##3}%
  \crefname{appendix}{付録}{付録}%
  \crefformat{appendix}{##2付録##1##3}%
  \crefname{equation}{式}{式}%
  \crefformat{equation}{##2式##1##3}%
  \cref{#1}%
  \endgroup
}
% ****************************************************************************************************


% ****************************************************************************************************
% 3. Loading some handy packages
% ****************************************************************************************************
% ********************************************************************
% Packages with options that might require adjustments
% ********************************************************************
\PassOptionsToPackage{ngerman,american}{babel} % change this to your language(s), main language last
% Spanish languages need extra options in order to work with this template
%\PassOptionsToPackage{spanish,es-lcroman}{babel}
    \usepackage{babel}

\usepackage{csquotes}
\PassOptionsToPackage{%
  backend=biber,%
  bibencoding=utf8,%
  language=auto,%
  style=apa,%
  sorting=nyt,%
  maxbibnames=20,%
  giveninits=false%
}{biblatex}
    \usepackage{biblatex}
    \setlength{\bibhang}{2em} % Hanging indentation of 2 Japanese characters
    
    % Japanese Paragraph Indentation
    \usepackage{indentfirst} % Indent first paragraph after section header
    \AtBeginDocument{\setlength{\parindent}{1em}} % Indent width: 1 full-width character
    
    % Fix for bibliography wrapping issues with justification
    \appto{\bibsetup}{\setlength{\emergencystretch}{3em}}

\DeclareLanguageMapping{american}{american-apa}
\DeclareLanguageMapping{english}{english-apa}
\DeclareLanguageMapping{japanese}{english-apa} % Use English APA as base for Japanese


% ********************************************************************
% Custom sorting: use yomi field for Japanese entries
% ********************************************************************
% Map yomi to sortname for Japanese entries only
% Biber accepts custom fields in sourcemaps without declaration
\DeclareSourcemap{
  \maps[datatype=bibtex]{
    \map[overwrite]{
      \step[fieldsource=langid, match=japanese, final]
      \step[fieldsource=yomi, fieldset=sortname, origfieldval]
    }
    \map[overwrite]{
      \step[typesource=incollectionja, typetarget=incollection, final]
      \step[fieldset=keywords, fieldvalue=incollectionja, append]
      \step[fieldset=langid, fieldvalue=japanese]
    }
  }
}

% ********************************************************************
% Japanese bibliography formatting
% ********************************************************************
% For Japanese entries, use:
%   author = {細田, 衛士 and 山本, 雅資}
%   langid = {japanese}
%   yomi = {ほそだ, えいじ and やまもと, まさし}
% Result: 細田衛士・山本雅資 (full name, no comma, ・ separator)
% Sorting: Uses hiragana yomi field for alphabetical ordering

% Change fixed labels for Japanese references
% Change fixed labels for Japanese references
\DefineBibliographyStrings{japanese}{
  in      = {},   % suppress the leading “In”
  editor  = {編}, % editor -> 編
  editors = {編}, % editors -> 編
}

% Override "In" macro
\renewbibmacro*{in}{%
  \ifboolexpr{
    test {\iffieldequalstr{langid}{japanese}}
    or
    test {\ifkeyword{incollectionja}}
  }
    {}
    {\printtext{\bibstring{in}\intitlepunct}}%
}

% Override editor string macro
\renewbibmacro*{apaeditorstrg}[1]{%
  \ifboolexpr{
    test {\iffieldequalstr{langid}{japanese}}
    or
    test {\ifkeyword{incollectionja}}
  }
    {編}
    {\iffieldundef{#1type}
       {\ifthenelse{\value{#1}>1\OR\ifandothers{#1}}
          {\bibcpstring{editors}}
          {\bibcpstring{editor}}}
       {\ifthenelse{\value{#1}>1\OR\ifandothers{#1}}
            {\bibcpstring{type\thefield{#1type}s}}
            {\bibcpstring{type\thefield{#1type}}}}}%
}

% Override the low-level name format used by APA
\DeclareNameFormat{apaauthor}{%
  \iffieldequalstr{langid}{japanese}
    {% Japanese: Family + Given, no comma, no abbreviation
      \namepartfamily
      \namepartgiven
      % Add ・ delimiter between names, but not after the last one
      \ifnumequal{\value{listcount}}{\value{liststop}}
        {}
        {・}%
    }
    {% Non-Japanese: standard APA format
      \ifcase\value{uniquename}%
        \usebibmacro{name:family-given}
          {\namepartfamily}
          {\namepartgiveni}
          {\namepartprefix}
          {\namepartsuffix}%
      \or
        \ifuseprefix
          {\usebibmacro{name:given-family}
            {\namepartfamily}
            {\namepartgiveni}
            {\namepartprefix}
            {\namepartsuffix}}
          {\usebibmacro{name:given-family}
            {\namepartfamily}
            {\namepartgiveni}
            {\namepartprefix}
            {\namepartsuffix}}%
      \or
        \usebibmacro{name:family-given}
          {\namepartfamily}
          {\namepartgiveni}
          {\namepartprefix}
          {\namepartsuffix}%
      \fi
    }%
  \usebibmacro{name:andothers}%
}

% Use the same format for editors
\DeclareNameAlias{apaeditor}{apaauthor}
\DeclareNameAlias{apatranslator}{apaauthor}
\DeclareNameAlias{apanames}{apaauthor}

% Also apply to citations (labelname is used for \cite, \parencite, etc.)
\DeclareNameFormat{labelname}{%
  \iffieldequalstr{langid}{japanese}
    {% Japanese: Family only for citations, with ・ delimiter
      \namepartfamily
      \ifnumequal{\value{listcount}}{\value{liststop}}
        {}
        {・}%
    }
    {% Non-Japanese: standard APA citation format
      \ifcase\value{uniquename}%
        \usebibmacro{name:family}
          {\namepartfamily}
          {\namepartgiven}
          {\namepartprefix}
          {\namepartsuffix}%
      \or
        \ifuseprefix
          {\usebibmacro{name:given-family}
            {\namepartfamily}
            {\namepartgiveni}
            {\namepartprefix}
            {\namepartsuffixi}}
          {\usebibmacro{name:given-family}
            {\namepartfamily}
            {\namepartgiveni}
            {\namepartprefix}
            {\namepartsuffixi}}%
      \or
        \usebibmacro{name:given-family}
          {\namepartfamily}
          {\namepartgiven}
          {\namepartprefix}
          {\namepartsuffix}%
      \fi
    }%
  \usebibmacro{name:andothers}%
}

% Override delimiters for Japanese entries globally
\DeclareDelimFormat{multinamedelim}{%
  \iffieldequalstr{langid}{japanese}
    {・}%
    {\addcomma\space}%
}

\DeclareDelimFormat{finalnamedelim}{%
  \iffieldequalstr{langid}{japanese}
    {・}%
    {\space\&\space}%
}

% Override "et al." to show "他" for Japanese entries
\renewbibmacro*{name:andothers}{%
  \ifboolexpr{
    test {\ifnumequal{\value{listcount}}{\value{liststop}}}
    and
    test \ifmorenames
  }
    {%
      \iffieldequalstr{langid}{japanese}
        {他}% Japanese: use 他 instead of et al.
        {\printdelim{andothersdelim}\bibstring{andothers}}% Non-Japanese: use et al.
    }
    {}%
}

% Disable italics for Japanese titles
\DeclareFieldFormat[article,book,inbook,incollection,inproceedings,report,thesis,unpublished]{title}{%
  \iffieldequalstr{langid}{japanese}
    {#1}% Japanese: no italics
    {\mkbibemph{#1}}% Non-Japanese: use italics
}

\DeclareFieldFormat[article,book,inbook,incollection,inproceedings,report,thesis,unpublished]{journaltitle}{%
  \iffieldequalstr{langid}{japanese}
    {#1}% Japanese: no italics
    {\mkbibemph{#1}}% Non-Japanese: use italics
}

\DeclareFieldFormat[article,book,inbook,incollection,inproceedings,report,thesis,unpublished]{booktitle}{%
  \iffieldequalstr{langid}{japanese}
    {#1}% Japanese: no italics
    {\mkbibemph{#1}}% Non-Japanese: use italics
}


\PassOptionsToPackage{fleqn}{amsmath}       % math environments and more by the AMS
  \usepackage{amsmath}

% ********************************************************************
% General useful packages
% ********************************************************************
\usepackage{graphicx} %
\usepackage{scrhack} % fix warnings when using KOMA with listings package
\usepackage{xspace} % to get the spacing after macros right
\PassOptionsToPackage{printonlyused,smaller}{acronym}
  \usepackage{acronym} % nice macros for handling all acronyms in the thesis
  %\renewcommand{\bflabel}[1]{{#1}\hfill} % fix the list of acronyms --> no longer working
  %\renewcommand*{\acsfont}[1]{\textsc{#1}}
  %\renewcommand*{\aclabelfont}[1]{\acsfont{#1}}
  %\def\bflabel#1{{#1\hfill}}
  \def\bflabel#1{{\acsfont{#1}\hfill}}
  \def\aclabelfont#1{\acsfont{#1}}
% ****************************************************************************************************
%\usepackage{pgfplots} % External TikZ/PGF support (thanks to Andreas Nautsch)
%\usetikzlibrary{external}
%\tikzexternalize[mode=list and make, prefix=ext-tikz/]
% ****************************************************************************************************


% ****************************************************************************************************
% 4. Setup floats: tables, (sub)figures, and captions
% ****************************************************************************************************
\usepackage{tabularx} % better tables
  \setlength{\extrarowheight}{3pt} % increase table row height
\newcommand{\tableheadline}[1]{\multicolumn{1}{l}{\spacedlowsmallcaps{#1}}}
\newcommand{\myfloatalign}{\centering} % to be used with each float for alignment
\usepackage{subfig}
% ****************************************************************************************************


% ****************************************************************************************************
% 5. Setup code listings
% ****************************************************************************************************
\usepackage{listings}
%\lstset{emph={trueIndex,root},emphstyle=\color{BlueViolet}}%\underbar} % for special keywords
\lstset{language=[LaTeX]Tex,%C++,
  morekeywords={PassOptionsToPackage,selectlanguage},
  keywordstyle=\color{RoyalBlue},%\bfseries,
  basicstyle=\small\ttfamily,
  %identifierstyle=\color{NavyBlue},
  commentstyle=\color{Green}\ttfamily,
  stringstyle=\rmfamily,
  numbers=none,%left,%
  numberstyle=\scriptsize,%\tiny
  stepnumber=5,
  numbersep=8pt,
  showstringspaces=false,
  breaklines=true,
  %frameround=ftff,
  %frame=single,
  belowcaptionskip=.75\baselineskip
  %frame=L
}
% ****************************************************************************************************


% ****************************************************************************************************
% 6. Last calls before the bar closes
% ****************************************************************************************************
% ********************************************************************
% Her Majesty herself
% ********************************************************************
\PassOptionsToPackage{hyperfootnotes=false}{hyperref}
\usepackage{classicthesis}


% ********************************************************************
% Fine-tune hyperreferences (hyperref should be called last)
% ********************************************************************
\hypersetup{%
  %draft, % hyperref's draft mode, for printing see below
  colorlinks=true, linktocpage=true, pdfstartpage=3, pdfstartview=FitV,%
  % uncomment the following line if you want to have black links (e.g., for printing)
  %colorlinks=false, linktocpage=false, pdfstartpage=3, pdfstartview=FitV, pdfborder={0 0 0},%
  breaklinks=true, pageanchor=true,%
  pdfpagemode=UseNone, %
  % pdfpagemode=UseOutlines,%
  plainpages=false, bookmarksnumbered, bookmarksopen=true, bookmarksopenlevel=1,%
  hypertexnames=true, pdfhighlight=/O,%nesting=true,%frenchlinks,%
  urlcolor=CTurl, linkcolor=CTlink, citecolor=CTcitation, %pagecolor=RoyalBlue,%
  %urlcolor=Black, linkcolor=Black, citecolor=Black, %pagecolor=Black,%
  pdftitle={\myTitle},%
  pdfauthor={\textcopyright\ \myName, \myUni, \myFaculty},%
  pdfsubject={},%
  pdfkeywords={},%
  pdfcreator={pdfLaTeX},%
  pdfproducer={LaTeX with hyperref and classicthesis}%
}
% ********************************************************************
% Setup autoreferences (hyperref and babel)
% ********************************************************************
% There are some issues regarding autorefnames
% http://www.tex.ac.uk/cgi-bin/texfaq2html?label=latexwords
% you have to redefine the macros for the
% language you use, e.g., american, ngerman
% (as chosen when loading babel/AtBeginDocument)
% ********************************************************************
 \makeatletter
 \@ifpackageloaded{babel}%
   {%
     \addto\extrasamerican{%
       \renewcommand*{\figureautorefname}{Figure}%
       \renewcommand*{\tableautorefname}{Table}%
       \renewcommand*{\partautorefname}{Part}%
       \renewcommand*{\chapterautorefname}{Chapter}%
       \renewcommand*{\sectionautorefname}{Section}%
       \renewcommand*{\subsectionautorefname}{Section}%
       \renewcommand*{\subsubsectionautorefname}{Section}%
       \renewcommand*{\figurename}{図}%
       \renewcommand*{\tablename}{表}%
     }%
     \addto\extrasngerman{%
       \renewcommand*{\paragraphautorefname}{Absatz}%
       \renewcommand*{\subparagraphautorefname}{Unterabsatz}%
       \renewcommand*{\footnoteautorefname}{Fu\"snote}%
       \renewcommand*{\FancyVerbLineautorefname}{Zeile}%
       \renewcommand*{\theoremautorefname}{Theorem}%
       \renewcommand*{\appendixautorefname}{Anhang}%
       \renewcommand*{\equationautorefname}{Gleichung}%
       \renewcommand*{\itemautorefname}{Punkt}%
     }%
       % Fix to getting autorefs for subfigures right (thanks to Belinda Vogt for changing the definition)
       \providecommand{\subfigureautorefname}{\figureautorefname}%
     }{\relax}
 \makeatother

% (Better) alternative to \autoref is \cref via the cleveref package
\usepackage{cleveref}
%\crefformat{part}{Part #2\MakeUppercase{#1}#3}


% ********************************************************************
% Development Stuff
% ********************************************************************
\listfiles
%\PassOptionsToPackage{l2tabu,orthodox,abort}{nag}
%  \usepackage{nag}
%\PassOptionsToPackage{warning, all}{onlyamsmath}
%  \usepackage{onlyamsmath}


% ****************************************************************************************************
% 7. Further adjustments (experimental)
% ****************************************************************************************************
% ********************************************************************
% Changing the text area
% ********************************************************************
\areaset[current]{370pt}{761pt} % Increased width from default 336pt
%\setlength{\marginparwidth}{7em}%
%\setlength{\marginparsep}{2em}%

% ********************************************************************
% Using different fonts
% ********************************************************************
%\usepackage[oldstylenums]{kpfonts} % oldstyle notextcomp
% \usepackage[osf]{libertine}
%\usepackage[light,condensed,math]{iwona}
%\renewcommand{\sfdefault}{iwona}
%\usepackage{lmodern} % <-- no osf support :-(
%\usepackage{cfr-lm} %
%\usepackage[urw-garamond]{mathdesign} <-- no osf support :-(
%\usepackage[default,osfigures]{opensans} % scale=0.95
%\usepackage[sfdefault]{FiraSans}
%\usepackage[opticals,mathlf]{MinionPro} % onlytext
% ********************************************************************
%\setmainfont[Ligatures=TeX]{Times New Roman} % Set Times New Roman as main font
% ********************************************************************
%\usepackage[largesc,osf]{newpxtext}
%\linespread{1.05} % a bit more for Palatino
% Used to fix these:
% https://bitbucket.org/amiede/classicthesis/issues/139/italics-in-pallatino-capitals-chapter
% https://bitbucket.org/amiede/classicthesis/issues/45/problema-testatine-su-classicthesis-style
% ********************************************************************
% ****************************************************************************************************

% ****************************************************************************************************
% If you like the classicthesis, then I would appreciate a postcard.
% My address can be found in the file ClassicThesis.pdf. A collection
% of the postcards I received so far is available online at
% http://postcards.miede.de
% ****************************************************************************************************


% ****************************************************************************************************
% 0. Set the encoding of your files. UTF-8 is the only sensible encoding nowadays. If you can't read
% äöüßáéçèê∂åëæƒÏ€ then change the encoding setting in your editor, not the line below. If your editor
% does not support utf8 use another editor!
% ****************************************************************************************************
\usepackage{iftex}
\ifPDFTeX
  \PassOptionsToPackage{utf8}{inputenc}
  \usepackage{inputenc}
  \PassOptionsToPackage{T1}{fontenc} % T2A for cyrillics
  \usepackage{fontenc}
\else
  \usepackage{fontspec}
\fi


% ****************************************************************************************************
% 1. Configure classicthesis for your needs here, e.g., remove "drafting" below
% in order to deactivate the time-stamp on the pages
% (see ClassicThesis.pdf for more information):
% ****************************************************************************************************
\PassOptionsToPackage{
  drafting=false,    % print version information on the bottom of the pages
  tocaligned=false, % the left column of the toc will be aligned (no indentation)
  dottedtoc=false,  % page numbers in ToC flushed right
  eulerchapternumbers=false, % use AMS Euler for chapter font (otherwise Palatino)
  floatperchapter=true,     % numbering per chapter for all floats (i.e., Figure 1.1)
  eulermath=false,  % use awesome Euler fonts for mathematical formulae (only with pdfLaTeX)
  beramono=true,    % toggle a nice monospaced font (w/ bold)
  palatino=false,    % deactivate standard font for loading another one, see the last section at the end of this file for suggestions
  %linedheaders=true, % obsolete / available for backwards compatibility
  style=classicthesis % classicthesis, arsclassica, linedheaders, plain
}{classicthesis}

% ****************************************************************************************************
% 2. Personal data and user ad-hoc commands (insert your own data here)
% ****************************************************************************************************
% ****************************************************************************************************
% 2. Personal data and user ad-hoc commands (insert your own data here)
% ****************************************************************************************************
\newcommand{\myTitle}{The Title of the Thesis\xspace}
\newcommand{\mySubtitle}{The Subtitle of the Thesis\xspace}
\newcommand{\myDegree}{Your Degree\xspace}
\newcommand{\myName}{Your Name\xspace}
\newcommand{\myProf}{Your Supervisor\xspace}
\newcommand{\myOtherProf}{Your Other Supervisor\xspace}
\newcommand{\mySupervisor}{Your Supervisor\xspace}
\newcommand{\myFaculty}{Your Faculty\xspace}
\newcommand{\myDepartment}{Your Department\xspace}
\newcommand{\myUni}{Your Uni\xspace}
\newcommand{\myLocation}{Location\xspace}
\newcommand{\myTime}{Month Year\xspace}
\newcommand{\myVersion}{v1.0\xspace}

% ********************************************************************
% Setup, finetuning, and useful commands
% ********************************************************************
\providecommand{\mLyX}{L\kern-.1667em\lower.25em\hbox{Y}\kern-.125emX\@}
\newcommand{\ie}{i.\,e.}
\newcommand{\Ie}{I.\,e.}
\newcommand{\eg}{e.\,g.}
\newcommand{\Eg}{E.\,g.}
\newcommand{\autorefja}[1]{%
  \begingroup
  \crefname{chapter}{第}{第}%
  \crefformat{chapter}{##2第##1章##3}%
  \crefname{section}{第}{第}%
  \crefformat{section}{##2第##1節##3}%
  \crefname{subsection}{第}{第}%
  \crefformat{subsection}{##2第##1項##3}%
  \crefname{figure}{図}{図}%
  \crefformat{figure}{##2図##1##3}%
  \crefname{table}{表}{表}%
  \crefformat{table}{##2表##1##3}%
  \crefname{appendix}{付録}{付録}%
  \crefformat{appendix}{##2付録##1##3}%
  \crefname{equation}{式}{式}%
  \crefformat{equation}{##2式##1##3}%
  \cref{#1}%
  \endgroup
}
% ****************************************************************************************************


% ****************************************************************************************************
% 3. Loading some handy packages
% ****************************************************************************************************
% ********************************************************************
% Packages with options that might require adjustments
% ********************************************************************
\PassOptionsToPackage{ngerman,american}{babel} % change this to your language(s), main language last
% Spanish languages need extra options in order to work with this template
%\PassOptionsToPackage{spanish,es-lcroman}{babel}
    \usepackage{babel}

\usepackage{csquotes}
\PassOptionsToPackage{%
  backend=biber,%
  bibencoding=utf8,%
  language=auto,%
  style=apa,%
  sorting=nyt,%
  maxbibnames=20,%
  giveninits=false%
}{biblatex}
    \usepackage{biblatex}
    \setlength{\bibhang}{2em} % Hanging indentation of 2 Japanese characters
    
    % Japanese Paragraph Indentation
    \usepackage{indentfirst} % Indent first paragraph after section header
    \AtBeginDocument{\setlength{\parindent}{1em}} % Indent width: 1 full-width character
    
    % Fix for bibliography wrapping issues with justification
    \appto{\bibsetup}{\setlength{\emergencystretch}{3em}}

\DeclareLanguageMapping{american}{american-apa}
\DeclareLanguageMapping{english}{english-apa}
\DeclareLanguageMapping{japanese}{english-apa} % Use English APA as base for Japanese


% ********************************************************************
% Custom sorting: use yomi field for Japanese entries
% ********************************************************************
% Map yomi to sortname for Japanese entries only
% Biber accepts custom fields in sourcemaps without declaration
\DeclareSourcemap{
  \maps[datatype=bibtex]{
    \map[overwrite]{
      \step[fieldsource=langid, match=japanese, final]
      \step[fieldsource=yomi, fieldset=sortname, origfieldval]
    }
    \map[overwrite]{
      \step[typesource=incollectionja, typetarget=incollection, final]
      \step[fieldset=keywords, fieldvalue=incollectionja, append]
      \step[fieldset=langid, fieldvalue=japanese]
    }
  }
}

% ********************************************************************
% Japanese bibliography formatting
% ********************************************************************
% For Japanese entries, use:
%   author = {細田, 衛士 and 山本, 雅資}
%   langid = {japanese}
%   yomi = {ほそだ, えいじ and やまもと, まさし}
% Result: 細田衛士・山本雅資 (full name, no comma, ・ separator)
% Sorting: Uses hiragana yomi field for alphabetical ordering

% Change fixed labels for Japanese references
% Change fixed labels for Japanese references
\DefineBibliographyStrings{japanese}{
  in      = {},   % suppress the leading “In”
  editor  = {編}, % editor -> 編
  editors = {編}, % editors -> 編
}

% Override "In" macro
\renewbibmacro*{in}{%
  \ifboolexpr{
    test {\iffieldequalstr{langid}{japanese}}
    or
    test {\ifkeyword{incollectionja}}
  }
    {}
    {\printtext{\bibstring{in}\intitlepunct}}%
}

% Override editor string macro
\renewbibmacro*{apaeditorstrg}[1]{%
  \ifboolexpr{
    test {\iffieldequalstr{langid}{japanese}}
    or
    test {\ifkeyword{incollectionja}}
  }
    {編}
    {\iffieldundef{#1type}
       {\ifthenelse{\value{#1}>1\OR\ifandothers{#1}}
          {\bibcpstring{editors}}
          {\bibcpstring{editor}}}
       {\ifthenelse{\value{#1}>1\OR\ifandothers{#1}}
            {\bibcpstring{type\thefield{#1type}s}}
            {\bibcpstring{type\thefield{#1type}}}}}%
}

% Override the low-level name format used by APA
\DeclareNameFormat{apaauthor}{%
  \iffieldequalstr{langid}{japanese}
    {% Japanese: Family + Given, no comma, no abbreviation
      \namepartfamily
      \namepartgiven
      % Add ・ delimiter between names, but not after the last one
      \ifnumequal{\value{listcount}}{\value{liststop}}
        {}
        {・}%
    }
    {% Non-Japanese: standard APA format
      \ifcase\value{uniquename}%
        \usebibmacro{name:family-given}
          {\namepartfamily}
          {\namepartgiveni}
          {\namepartprefix}
          {\namepartsuffix}%
      \or
        \ifuseprefix
          {\usebibmacro{name:given-family}
            {\namepartfamily}
            {\namepartgiveni}
            {\namepartprefix}
            {\namepartsuffix}}
          {\usebibmacro{name:given-family}
            {\namepartfamily}
            {\namepartgiveni}
            {\namepartprefix}
            {\namepartsuffix}}%
      \or
        \usebibmacro{name:family-given}
          {\namepartfamily}
          {\namepartgiveni}
          {\namepartprefix}
          {\namepartsuffix}%
      \fi
    }%
  \usebibmacro{name:andothers}%
}

% Use the same format for editors
\DeclareNameAlias{apaeditor}{apaauthor}
\DeclareNameAlias{apatranslator}{apaauthor}
\DeclareNameAlias{apanames}{apaauthor}

% Also apply to citations (labelname is used for \cite, \parencite, etc.)
\DeclareNameFormat{labelname}{%
  \iffieldequalstr{langid}{japanese}
    {% Japanese: Family only for citations, with ・ delimiter
      \namepartfamily
      \ifnumequal{\value{listcount}}{\value{liststop}}
        {}
        {・}%
    }
    {% Non-Japanese: standard APA citation format
      \ifcase\value{uniquename}%
        \usebibmacro{name:family}
          {\namepartfamily}
          {\namepartgiven}
          {\namepartprefix}
          {\namepartsuffix}%
      \or
        \ifuseprefix
          {\usebibmacro{name:given-family}
            {\namepartfamily}
            {\namepartgiveni}
            {\namepartprefix}
            {\namepartsuffixi}}
          {\usebibmacro{name:given-family}
            {\namepartfamily}
            {\namepartgiveni}
            {\namepartprefix}
            {\namepartsuffixi}}%
      \or
        \usebibmacro{name:given-family}
          {\namepartfamily}
          {\namepartgiven}
          {\namepartprefix}
          {\namepartsuffix}%
      \fi
    }%
  \usebibmacro{name:andothers}%
}

% Override delimiters for Japanese entries globally
\DeclareDelimFormat{multinamedelim}{%
  \iffieldequalstr{langid}{japanese}
    {・}%
    {\addcomma\space}%
}

\DeclareDelimFormat{finalnamedelim}{%
  \iffieldequalstr{langid}{japanese}
    {・}%
    {\space\&\space}%
}

% Override "et al." to show "他" for Japanese entries
\renewbibmacro*{name:andothers}{%
  \ifboolexpr{
    test {\ifnumequal{\value{listcount}}{\value{liststop}}}
    and
    test \ifmorenames
  }
    {%
      \iffieldequalstr{langid}{japanese}
        {他}% Japanese: use 他 instead of et al.
        {\printdelim{andothersdelim}\bibstring{andothers}}% Non-Japanese: use et al.
    }
    {}%
}

% Disable italics for Japanese titles
\DeclareFieldFormat[article,book,inbook,incollection,inproceedings,report,thesis,unpublished]{title}{%
  \iffieldequalstr{langid}{japanese}
    {#1}% Japanese: no italics
    {\mkbibemph{#1}}% Non-Japanese: use italics
}

\DeclareFieldFormat[article,book,inbook,incollection,inproceedings,report,thesis,unpublished]{journaltitle}{%
  \iffieldequalstr{langid}{japanese}
    {#1}% Japanese: no italics
    {\mkbibemph{#1}}% Non-Japanese: use italics
}

\DeclareFieldFormat[article,book,inbook,incollection,inproceedings,report,thesis,unpublished]{booktitle}{%
  \iffieldequalstr{langid}{japanese}
    {#1}% Japanese: no italics
    {\mkbibemph{#1}}% Non-Japanese: use italics
}


\PassOptionsToPackage{fleqn}{amsmath}       % math environments and more by the AMS
  \usepackage{amsmath}

% ********************************************************************
% General useful packages
% ********************************************************************
\usepackage{graphicx} %
\usepackage{scrhack} % fix warnings when using KOMA with listings package
\usepackage{xspace} % to get the spacing after macros right
\PassOptionsToPackage{printonlyused,smaller}{acronym}
  \usepackage{acronym} % nice macros for handling all acronyms in the thesis
  %\renewcommand{\bflabel}[1]{{#1}\hfill} % fix the list of acronyms --> no longer working
  %\renewcommand*{\acsfont}[1]{\textsc{#1}}
  %\renewcommand*{\aclabelfont}[1]{\acsfont{#1}}
  %\def\bflabel#1{{#1\hfill}}
  \def\bflabel#1{{\acsfont{#1}\hfill}}
  \def\aclabelfont#1{\acsfont{#1}}
% ****************************************************************************************************
%\usepackage{pgfplots} % External TikZ/PGF support (thanks to Andreas Nautsch)
%\usetikzlibrary{external}
%\tikzexternalize[mode=list and make, prefix=ext-tikz/]
% ****************************************************************************************************


% ****************************************************************************************************
% 4. Setup floats: tables, (sub)figures, and captions
% ****************************************************************************************************
\usepackage{tabularx} % better tables
  \setlength{\extrarowheight}{3pt} % increase table row height
\newcommand{\tableheadline}[1]{\multicolumn{1}{l}{\spacedlowsmallcaps{#1}}}
\newcommand{\myfloatalign}{\centering} % to be used with each float for alignment
\usepackage{subfig}
% ****************************************************************************************************


% ****************************************************************************************************
% 5. Setup code listings
% ****************************************************************************************************
\usepackage{listings}
%\lstset{emph={trueIndex,root},emphstyle=\color{BlueViolet}}%\underbar} % for special keywords
\lstset{language=[LaTeX]Tex,%C++,
  morekeywords={PassOptionsToPackage,selectlanguage},
  keywordstyle=\color{RoyalBlue},%\bfseries,
  basicstyle=\small\ttfamily,
  %identifierstyle=\color{NavyBlue},
  commentstyle=\color{Green}\ttfamily,
  stringstyle=\rmfamily,
  numbers=none,%left,%
  numberstyle=\scriptsize,%\tiny
  stepnumber=5,
  numbersep=8pt,
  showstringspaces=false,
  breaklines=true,
  %frameround=ftff,
  %frame=single,
  belowcaptionskip=.75\baselineskip
  %frame=L
}
% ****************************************************************************************************


% ****************************************************************************************************
% 6. Last calls before the bar closes
% ****************************************************************************************************
% ********************************************************************
% Her Majesty herself
% ********************************************************************
\PassOptionsToPackage{hyperfootnotes=false}{hyperref}
\usepackage{classicthesis}


% ********************************************************************
% Fine-tune hyperreferences (hyperref should be called last)
% ********************************************************************
\hypersetup{%
  %draft, % hyperref's draft mode, for printing see below
  colorlinks=true, linktocpage=true, pdfstartpage=3, pdfstartview=FitV,%
  % uncomment the following line if you want to have black links (e.g., for printing)
  %colorlinks=false, linktocpage=false, pdfstartpage=3, pdfstartview=FitV, pdfborder={0 0 0},%
  breaklinks=true, pageanchor=true,%
  pdfpagemode=UseNone, %
  % pdfpagemode=UseOutlines,%
  plainpages=false, bookmarksnumbered, bookmarksopen=true, bookmarksopenlevel=1,%
  hypertexnames=true, pdfhighlight=/O,%nesting=true,%frenchlinks,%
  urlcolor=CTurl, linkcolor=CTlink, citecolor=CTcitation, %pagecolor=RoyalBlue,%
  %urlcolor=Black, linkcolor=Black, citecolor=Black, %pagecolor=Black,%
  pdftitle={\myTitle},%
  pdfauthor={\textcopyright\ \myName, \myUni, \myFaculty},%
  pdfsubject={},%
  pdfkeywords={},%
  pdfcreator={pdfLaTeX},%
  pdfproducer={LaTeX with hyperref and classicthesis}%
}
% ********************************************************************
% Setup autoreferences (hyperref and babel)
% ********************************************************************
% There are some issues regarding autorefnames
% http://www.tex.ac.uk/cgi-bin/texfaq2html?label=latexwords
% you have to redefine the macros for the
% language you use, e.g., american, ngerman
% (as chosen when loading babel/AtBeginDocument)
% ********************************************************************
 \makeatletter
 \@ifpackageloaded{babel}%
   {%
     \addto\extrasamerican{%
       \renewcommand*{\figureautorefname}{Figure}%
       \renewcommand*{\tableautorefname}{Table}%
       \renewcommand*{\partautorefname}{Part}%
       \renewcommand*{\chapterautorefname}{Chapter}%
       \renewcommand*{\sectionautorefname}{Section}%
       \renewcommand*{\subsectionautorefname}{Section}%
       \renewcommand*{\subsubsectionautorefname}{Section}%
       \renewcommand*{\figurename}{図}%
       \renewcommand*{\tablename}{表}%
     }%
     \addto\extrasngerman{%
       \renewcommand*{\paragraphautorefname}{Absatz}%
       \renewcommand*{\subparagraphautorefname}{Unterabsatz}%
       \renewcommand*{\footnoteautorefname}{Fu\"snote}%
       \renewcommand*{\FancyVerbLineautorefname}{Zeile}%
       \renewcommand*{\theoremautorefname}{Theorem}%
       \renewcommand*{\appendixautorefname}{Anhang}%
       \renewcommand*{\equationautorefname}{Gleichung}%
       \renewcommand*{\itemautorefname}{Punkt}%
     }%
       % Fix to getting autorefs for subfigures right (thanks to Belinda Vogt for changing the definition)
       \providecommand{\subfigureautorefname}{\figureautorefname}%
     }{\relax}
 \makeatother

% (Better) alternative to \autoref is \cref via the cleveref package
\usepackage{cleveref}
%\crefformat{part}{Part #2\MakeUppercase{#1}#3}


% ********************************************************************
% Development Stuff
% ********************************************************************
\listfiles
%\PassOptionsToPackage{l2tabu,orthodox,abort}{nag}
%  \usepackage{nag}
%\PassOptionsToPackage{warning, all}{onlyamsmath}
%  \usepackage{onlyamsmath}


% ****************************************************************************************************
% 7. Further adjustments (experimental)
% ****************************************************************************************************
% ********************************************************************
% Changing the text area
% ********************************************************************
\areaset[current]{370pt}{761pt} % Increased width from default 336pt
%\setlength{\marginparwidth}{7em}%
%\setlength{\marginparsep}{2em}%

% ********************************************************************
% Using different fonts
% ********************************************************************
%\usepackage[oldstylenums]{kpfonts} % oldstyle notextcomp
% \usepackage[osf]{libertine}
%\usepackage[light,condensed,math]{iwona}
%\renewcommand{\sfdefault}{iwona}
%\usepackage{lmodern} % <-- no osf support :-(
%\usepackage{cfr-lm} %
%\usepackage[urw-garamond]{mathdesign} <-- no osf support :-(
%\usepackage[default,osfigures]{opensans} % scale=0.95
%\usepackage[sfdefault]{FiraSans}
%\usepackage[opticals,mathlf]{MinionPro} % onlytext
% ********************************************************************
%\setmainfont[Ligatures=TeX]{Times New Roman} % Set Times New Roman as main font
% ********************************************************************
%\usepackage[largesc,osf]{newpxtext}
%\linespread{1.05} % a bit more for Palatino
% Used to fix these:
% https://bitbucket.org/amiede/classicthesis/issues/139/italics-in-pallatino-capitals-chapter
% https://bitbucket.org/amiede/classicthesis/issues/45/problema-testatine-su-classicthesis-style
% ********************************************************************
% ****************************************************************************************************
